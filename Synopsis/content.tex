\pdfbookmark{General characteristics of the work}{characteristic}             % Закладка pdf
\section*{\centerline{General characteristics of the work}}

\newcommand{\actuality}{\pdfbookmark[1]{Relevance of the work}{actuality}\underline{\textbf{\actualityTXT}}}
\newcommand{\aim}{\pdfbookmark[1]{Aim}{aim}\underline{{\textbf\aimTXT}}}
\newcommand{\tasks}{\pdfbookmark[1]{Tasks}{tasks}\underline{\textbf{\tasksTXT}}}
\newcommand{\novelty}{\pdfbookmark[1]{Scientific novelty}{novelty}\underline{\textbf{\noveltyTXT}}}
\newcommand{\influence}{\pdfbookmark[1]{Scientific and practical significance}{influence}\underline{\textbf{\influenceTXT}}}
\newcommand{\methods}{\pdfbookmark[1]{Research methodology}{methods}\underline{\textbf{\methodsTXT}}}
\newcommand{\defpositions}{\pdfbookmark[1]{Propositions submitted for the defense}{defpositions}\underline{\textbf{\defpositionsTXT}}}
\newcommand{\reliability}{\pdfbookmark[1]{Validity and reliability}{reliability}\underline{\textbf{\reliabilityTXT}}}
\newcommand{\probation}{\pdfbookmark[1]{Approbation of the results}{probation}\underline{\textbf{\probationTXT}}}
\newcommand{\contribution}{\pdfbookmark[1]{Personal contribution}{contribution}\underline{\textbf{\contributionTXT}}}
\newcommand{\publications}{\pdfbookmark[1]{Publications}{publications}\underline{\textbf{\publicationsTXT}}}

\newtheorem{theorem}{Theorem}
\newtheorem{lemma}[theorem]{Lemma}
\newtheorem{assumption}{Assumption}
{\actuality}
I highly recommend reading a guide by Professor Oleg V. Vasilyev on dissertation writing (added in the repository), especially before you start to write the Introduction part. The Introduction part in a VAK-alike dissertation has a rigorous structure, comprising a particular set of paragraphs. This part typically completely repeats the introductory part from the Synopsis. 


Below are the examples how to use the keywords for dissertation characteristics description. 

{\aim} is to develop blablabla. To achieve the stated aim, it is necessary to solve the following {\tasks}:
\begin{enumerate}[beginpenalty=10000] % https://tex.stackexchange.com/a/476052/104425
\item Problem 1. 
\end{enumerate}

{\novelty}
There is such a research gap, such a research gap, and one more research gap.  
The gaps mentioned above in the current research area stage lead to the claim of the following novelty of the results:
\begin{enumerate}[beginpenalty=10000] % https://tex.stackexchange.com/a/476052/104425
	\item I do cool things because it bridges this gap; 
\end{enumerate}

{\influence}
Bla Bla

{\methods}
Bla Bla

{\defpositions}
\begin{enumerate}[beginpenalty=10000] % https://tex.stackexchange.com/a/476052/104425
  \item The most important part of your dissertation. Each of the def-position should be clear, concise, measurable, and novel.  
\end{enumerate}

{\reliability} Bla Bla. 

{\probation}
The  results  and  main  provisions  of  the  work  were  presented  at  international 
conferences:
\begin{enumerate}
	\item Name of conference, Date;
\end{enumerate}

{\contribution} Bla Bla. 

% Just an example
{\publications} The dissertation materials were published in X works, including X papers in Q1 scientific journals indexed by Web~of Science and Scopus. In X publications, the author of the dissertation is the first author. 
 % Характеристика работы по структуре во введении и в автореферате не отличается (ГОСТ Р 7.0.11, пункты 5.3.1 и 9.2.1), потому её загружаем из одного и того же внешнего файла, предварительно задав форму выделения некоторым параметрам

%Диссертационная работа была выполнена при поддержке грантов \dots

%\underline{\textbf{Объем и структура работы.}} Диссертация состоит из~введения,
%четырех глав, заключения и~приложения. Полный объем диссертации
%\textbf{ХХХ}~страниц текста с~\textbf{ХХ}~рисунками и~5~таблицами. Список
%литературы содержит \textbf{ХХX}~наименование.

\pdfbookmark{Dissertation Summary}{description}                          % Закладка pdf
\section*{\centerline{Dissertation Summary}}
The main content of the synposis. It is better to keep to a number of pages that is a multiple of 4 (for printing brochures). The maximum number of pages if you use this template <<as it is>> is approximately 36 (one and a half author's sheets). 

\pdfbookmark{Conclusion}{concl}
\section*{\centerline{Conclusion}}
\input{common/concl}

\ifdefmacro{\microtypesetup}{\microtypesetup{protrusion=false}}{} % не рекомендуется применять пакет микротипографики к автоматически генерируемому списку литературы
\urlstyle{rm}                               % ссылки URL обычным шрифтом
\insertbibliofull
\ifdefmacro{\microtypesetup}{\microtypesetup{protrusion=true}}{}
\urlstyle{tt}                               % возвращаем установки шрифта ссылок URL

\newpage
\pdfbookmark{Author's publications on the dissertation topic}{publ}
\section*{\centerline{Author's publications on the dissertation topic}}
% The simplest way is to do it manually:) 
\begin{enumerate}[leftmargin=18pt]
	\item \textbf{Dissertation's author}, other authors, Paper's title, journal, year, DOI;
\end{enumerate}
