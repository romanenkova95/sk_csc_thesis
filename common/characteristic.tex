Below are the examples how to use the keywords for dissertation characteristics description. 

{\aim} is to develop blablabla. To achieve the stated aim, it is necessary to solve the following {\tasks}:
\begin{enumerate}[beginpenalty=10000] % https://tex.stackexchange.com/a/476052/104425
\item Problem 1. 
\end{enumerate}

{\novelty}
There is such a research gap, such a research gap, and one more research gap.  
The gaps mentioned above in the current research area stage lead to the claim of the following novelty of the results:
\begin{enumerate}[beginpenalty=10000] % https://tex.stackexchange.com/a/476052/104425
	\item I do cool things because it bridges this gap; 
\end{enumerate}

{\influence}
Bla Bla

{\methods}
Bla Bla

{\defpositions}
\begin{enumerate}[beginpenalty=10000] % https://tex.stackexchange.com/a/476052/104425
  \item The most important part of your dissertation. Each of the def-position should be clear, concise, measurable, and novel.  
\end{enumerate}

{\reliability} Bla Bla. 

{\probation}
The  results  and  main  provisions  of  the  work  were  presented  at  international 
conferences:
\begin{enumerate}
	\item Name of conference, Date;
\end{enumerate}

{\contribution} Bla Bla. 

% Just an example
{\publications} The dissertation materials were published in X works, including X papers in Q1 scientific journals indexed by Web~of Science and Scopus. In X publications, the author of the dissertation is the first author. 
