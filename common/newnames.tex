% Новые переменные, которые могут использоваться во всём проекте
% ГОСТ 7.0.11-2011
% 9.2 Оформление текста автореферата диссертации
% 9.2.1 Общая характеристика работы включает в себя следующие основные структурные
% элементы:
% актуальность темы исследования;
%TODO: при защите на русском языке, раскомментируйте соответствующие варианты
%\newcommand{\actualityTXT}{Актуальность темы исследования.}
\newcommand{\actualityTXT}{Relevance of the work.}
% степень ее разработанности;
%\newcommand{\progressTXT}{Степень разработанности темы.}
\newcommand{\progressTXT}{The degree of topic development.}
% цели и задачи;
%\newcommand{\aimTXT}{Цели научной работы}
% \newcommand{\tasksTXT}{Задачи научной работы}
\newcommand{\aimTXT}{The aim of the work}
\newcommand{\tasksTXT}{problems} % like the following problems are addressed
% научную новизну;
%\newcommand{\noveltyTXT}{Научная новизна.}
\newcommand{\noveltyTXT}{Scientific novelty.}
% теоретическую и практическую значимость работы;
%\newcommand{\influenceTXT}{Теоретическая и практическая значимость}
\newcommand{\influenceTXT}{Theoretical and practical significance.}
% или чаще используют просто
% методологию и методы исследования;
%\newcommand{\methodsTXT}{Методология и методы исследования.}
\newcommand{\methodsTXT}{Methodology and research methods.}
% положения, выносимые на защиту;
%\newcommand{\defpositionsTXT}{Положения, выносимые на защиту:}
\newcommand{\defpositionsTXT}{Propositions submitted for the defense:}
% or alternative version
%\newcommand{\defpositionsTXT}{Main results submitted for the defensе:}  
% степень достоверности и апробацию результатов.
%\newcommand{\reliabilityTXT}{Степень достоверности полученных результатов}
\newcommand{\reliabilityTXT}{Validity of the obtained results}
%\newcommand{\probationTXT}{Апробация результатов.}
\newcommand{\probationTXT}{Approbation of the results.}

%\newcommand{\contributionTXT}{Персональный вклад автора.}
\newcommand{\contributionTXT}{Personal contribution of the author.}

%\newcommand{\publicationsTXT}{Публикации по теме диссертации.}
\newcommand{\publicationsTXT}{Publications.}


% Автогенерация библиографии работает как-то косячно, поэтому местами я просто руками прописала
%%% Заголовки библиографии:

% для автореферата:
%\newcommand{\bibtitleauthor}{Публикации автора по теме диссертации}
\newcommand{\bibtitleauthor}{Publications of the Author on the Subject of the Dissertation}

% для стиля библиографии `\insertbiblioauthorgrouped`
\newcommand{\bibtitleauthorvak}{В изданиях из списка ВАК РФ}
\newcommand{\bibtitleauthorscopus}{В изданиях, входящих в международную базу цитирования Scopus}
\newcommand{\bibtitleauthorwos}{В изданиях, входящих в международную базу цитирования Web of Science}
\newcommand{\bibtitleauthorother}{В прочих изданиях}
\newcommand{\bibtitleauthorconf}{В сборниках трудов конференций}
\newcommand{\bibtitleauthorpatent}{Зарегистрированные патенты}
\newcommand{\bibtitleauthorprogram}{Зарегистрированные программы для ЭВМ}

% для стиля библиографии `\insertbiblioauthorimportant`:
\newcommand{\bibtitleauthorimportant}{Наиболее значимые \protect\MakeLowercase\bibtitleauthor}

% для списка литературы в диссертации и списка чужих работ в автореферате:
%\newcommand{\bibtitlefull}{Список литературы} % (ГОСТ Р 7.0.11-2011, 4)
\newcommand{\bibtitlefull}{References} % (ГОСТ Р 7.0.11-2011, 4)
