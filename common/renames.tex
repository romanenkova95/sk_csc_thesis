%%% Переопределение именований %%%
%TODO: если вы защищаетесь на русском языке, раскомментируйте соответствующие версии
%\renewcommand{\contentsname}{Оглавление}% (ГОСТ Р 7.0.11-2011, 4)
\renewcommand{\contentsname}{Content}
%\renewcommand{\figurename}{Рисунок}% (ГОСТ Р 7.0.11-2011, 5.3.9)
\renewcommand{\figurename}{Figure}
%\renewcommand{\tablename}{Таблица}% (ГОСТ Р 7.0.11-2011, 5.3.10)
\renewcommand{\tablename}{Table}

\renewcommand{\listfigurename}{Список рисунков}
\renewcommand{\listfigurename}{List of figures}
\renewcommand{\listtablename}{Список таблиц}
\renewcommand{\listtablename}{List of tables}
\renewcommand{\bibname}{\bibtitlefull}%
%\renewcommand{\nomname}{Список сокращений и условных обозначений}%
\renewcommand{\nomname}{List of Abbreviations}%
% TODO: раскомментировать для диссертации на русском языке
% Переопределения названий для nomencl. Так как опция russian не для utf8
%\renewcommand{\eqdeclaration}[1]{, see~(#1)}%
%\renewcommand{\pagedeclaration}[1]{, pg.~#1}%
%\renewcommand{\nomAname}{Латинские буквы}%
%\renewcommand{\nomGname}{Греческие буквы}%
%\renewcommand{\nomXname}{Верхние индексы}%
%\renewcommand{\nomZname}{Индексы}%
